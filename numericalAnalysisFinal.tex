%------------------------------------------------------------------------------
% Beginning of journal.tex
%------------------------------------------------------------------------------
%
% AMS-LaTeX version 2 sample file for journals, based on amsart.cls.
%
%        ***     DO NOT USE THIS FILE AS A STARTER.      ***
%        ***  USE THE JOURNAL-SPECIFIC *.TEMPLATE FILE.  ***
%
% Replace amsart by the documentclass for the target journal, e.g., tran-l.
%
\documentclass{amsart}
%     If your article includes graphics, uncomment this command.
\usepackage{amsthm,graphicx,amssymb,textcomp,cancel,bbm,enumerate}


\setcounter{section}{-1}

\newtheorem{theorem}{Theorem}%[section]
\newtheorem{lemma}[theorem]{Lemma}

\theoremstyle{definition}
\newtheorem{definition}{Definition}
\newtheorem{example}[theorem]{Example}
\newtheorem{xca}[theorem]{Exercise}

\theoremstyle{remark}
\newtheorem{remark}[theorem]{Remark}
\newtheorem*{remark*}{Remark}

\theoremstyle{definition}
\newtheorem{notation}[theorem]{Notation}

\numberwithin{equation}{section}
%\setcounter{section}{-1}
%%
\mathchardef\ordinarycolon\mathcode`\:
\mathcode`\:=\string"8000
\begingroup \catcode`\:=\active
  \gdef:{\mathrel{\mathop\ordinarycolon}}
\endgroup


\newcommand\nsleq{\mathrel{\ooalign{$\leqslant$\cr\hidewidth$|$\hidewidth\cr}}}
\newcommand{\sleq}{\leqslant}
\newcommand{\No}{\mathbb{N}\bf{o}}
\newcommand{\abs}[1]{\lvert#1\rvert}

%    Blank box placeholder for figures (to avoid requiring any
%    particular graphics capabilities for printing this document).
\newcommand{\blankbox}[2]{%
  \parbox{\columnwidth}{\centering
%    Set fboxsep to 0 so that the actual size of the box will match the
%    given measurements more closely.
    \setlength{\fboxsep}{0pt}%
    \fbox{\raisebox{0pt}[#2]{\hspace{#1}}}%
  }%
}

\begin{document}
\title{Numerical Analysis Programing Project \\ Dr. Songming Hou}
\author{John Emory}
\address{Program of Mathematics and Statistics, Louisiana Tech University}
\email{jfe004@latech.edu}
\date{\today}
\maketitle

\section{Introduction}
Tom the Cat is chasing Jerry the Mouse, with an initial gap between them of $100$m. Tom and Jerry's velocities
are given as $v_c = 4 - at$ ms$^{-1}$ and $v_m = v_{max}-ks = 3 - 0.02s$ ms$^{-1}$, respectively,
with $0<a$.
The velocity of the change in the gap between Tom and Jerry, $s$, is given by
$\frac{ds}{dt} = v_m - v_c = -1 -0.02s + at$ ms$^{-1}$.

\section{Problem}
Find the true solution for when Tom will catch Jerry by plotting the gap distance.
\\
\\
First, we need to solve $\frac{ds}{dt}$. Noting that our equation is a linear first-order ODE, we
need to put it into standard form:
\[\frac{ds}{dt} + 0.02s = at -1\]
Next, we find the integration factor. Observing that in the second additive term on the left hand side
we are multiplying by $t^0$, we see the integration factor is $e^{0.02t}$. This gives us the form:
\[\frac{d}{dt}s\cdot e^{0.02t} = ( at-1 )\cdot e^{0.02t}\]
Taking the antiderivative of both sides gives:
\[\int\frac{d}{dt}s\cdot e^{0.02t} dt = a\cdot \int t \cdot e^{0.02t} dt - \int e^{0.02t} dt\]
\[s\cdot e^{0.02t} = 50at\cdot e^{0.02t} = 2500a\cdot e^{0.02t} - 50e^{0.02t} +c\]
Then, canceling $e^{0.02t}$ gives:
\[s = 50a(t-50)-50+c \cdot e^{-0.02t}\]
Solving for $c$ at our initial value of $s(0)=100$ m will yield an equation we can use software to plot. Since $t=0$, we have:
\[100=-2500a-50+c \cdot e^{-0.02t}\]
\[c = 2500a + 150\]
So, our final equaiton we want to plot is:
\[s(a,t) = 50a(t-50+50\cdot e^{-0.02t})+150\cdot e^{-0.02t} -50\]
\section{Problem}
For $a = 0.01$ ms$^{-2}$, use the fourth-order Runge-Kutta method to compute when Tom will catch Jerry.
Use an appropriate step size to ensure an accurate result.

\section{Problem}
Use the Adams-Bashforth forth-order predictor-corrector to compute when Tom will catch Jerry using the
results form Runge-Kutta, above, for the initial values of Adams-Bashforth.

\section{Problem}
Suppose Tom's acceleration is unknown. If Tom does not catch Jerry in $120$s, is it possible that Tom
will catch Jerry?

\end{document}